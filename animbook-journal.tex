% !TeX root = RJwrapper.tex
\title{animbook: An R Package for Visualize \ldots{}}
\author{by Krisanat Anukarnsakulchularp}

\maketitle

\abstract{%
An abstract of less than 150 words.
}

\hypertarget{introduction}{%
\section{Introduction}\label{introduction}}

With the increasing computational powers, animation with large amounts of data is now become more accessible than ever. With the inspiration from a New York Times article, ``Extensive Data Shows Punishing Reach of Racism for Black Boys''. The animation portrays the trajectory of how boys with different demographics could land up in society. With this plot in mind, the paper proposed a new visualization to help capture the movement of the observation for each different group over the specific period.

This paper will focus on designing and validating the new visualization using a nested model (Munzner (2009)) to offer an animation tool for conveying the observation trend in the given period. The cognitive theory of multimedia learning (Mayer (2005)) will be considered during the designing period. This theory employs the structure and the strategies that will help the learner learn more effectively.

The challenge of the New York Times article animation was the amount of code needed to reproduce. Additionally, the business data may or may not have the same information as the one in that article. It establishes a need to create tools for generalizing the animation to a wider range of data while still maintaining the main objective.

The structure of this paper will consist of four main section. The first section is domain problem and data characterization, identifying the problems the visualization tools try to solve. Secondly, the operation and data type abstraction, the description of operations, and the types of data required for the next stage. The third section is visualization and interaction design, designing the visuals and refining the tools. In the last section, package design, develop the structure of the packages and how to implement the final visualize design into the R package. The checking of the threat and validating of that threat will be carried out throughout the entire process of designing the animbook package.

\hypertarget{domain-problem-and-data-characterization}{%
\section{Domain problem and data characterization}\label{domain-problem-and-data-characterization}}

\hypertarget{operation-and-data-type-abstraction}{%
\section{Operation and data type abstraction}\label{operation-and-data-type-abstraction}}

When working with the raw data, the data pre-processing is one of the important steps we need to take before further analysis is taken. The function in this package will offer some flexibility in the types of the variable it accepted. However, there are still some restrictions the user needs to follow. In this section, it will provide what are the accepted format and examples for the user the followed to reproduce the final animate plot.

The data set that will be used in this section and in all of the examples of this journal is collected from Bureau van Dijk. The data set contained 30,000 rows and 94 variable of information on listed, and major unlisted/delisted companies across the world. From the raw data, we are only interested in the ranking of the companies. The cleaned version of this data set is included in the package which only contained the relevant variables to reproduce the animated plot.

\hypertarget{visualization-and-interaction-design}{%
\section{Visualization and interaction design}\label{visualization-and-interaction-design}}

\hypertarget{package-design}{%
\section{Package design}\label{package-design}}

Introduction

Some ideas:

\begin{itemize}
\tightlist
\item
  The animated plot has not been used often in the business data presentation
\item
  We want to see the movement of our variable of interest
\item
  Business data: Accounting data (sales), Marketing data (customer interest), data analysis languages data
\item
  Help with storytelling
\item
  Inspiration: The New York Times (Extensive data shows punishing reach of racism for black boys). The animation plot shows what demographic group between white and black boys where would end up in society. It could be adapted to show, based on the demographic of the company, sees the movement of where they would end up. In the marketing data, the demographic of the customers could tell us the story of what product they preferred.
\end{itemize}

Data and Process

Some ideas:

\begin{itemize}
\item
  The data will need a following structure for it to be ready to plot: time index(time variable for the gganimate), x-axis for ggplot(this does not need to be the same as time index), key(unique identifier), rank(y-axis, factor variable), group(optional).
\item
  The user can input any numerical, categorical and factor variable into the prep\_anim function which will return the data in the format that the anim\_plot function accepted. The x-axis and time index does not need to be the same as this allows for the user to reproduce the nytimes plot which is not time dependent. For the rank variable, no matter what the type of variable is in, it will return numerical variable.
\end{itemize}

When working with the raw data, the data pre-processing is one of the important steps we need to take before further analysis is taken. The function in this package will offer some flexibility in the types of the variable it accepted. However, there are still some restrictions the user needs to follow. In this section, it will provide what are the accepted format and examples for the user the followed to reproduce the final animate plot.

The data set that will be used in this section and in all of the examples of this journal is collected from Bureau van Dijk. The data set contained 30,000 rows and 94 variable of information on listed, and major unlisted/delisted companies across the world. From the raw data, we are only interested in the ranking of the companies. The cleaned version of this data set is included in the package which only contained the relevant variables to reproduce the animated plot.

\begin{itemize}
\tightlist
\item
  The data structure
\item
  The steps for the user
\item
  How does the function processes the data
\end{itemize}

\hypertarget{references}{%
\section*{References}\label{references}}
\addcontentsline{toc}{section}{References}

\hypertarget{refs}{}
\begin{CSLReferences}{1}{0}
\leavevmode\vadjust pre{\hypertarget{ref-mayer_2005}{}}%
Mayer, Richard E. 2005. {``Cognitive Theory of Multimedia Learning.''} In \emph{The Cambridge Handbook of Multimedia Learning}, edited by RichardEditor Mayer, 31--48. Cambridge Handbooks in Psychology. Cambridge University Press. \url{https://doi.org/10.1017/CBO9780511816819.004}.

\leavevmode\vadjust pre{\hypertarget{ref-nested_model}{}}%
Munzner, Tamara. 2009. {``A Nested Model for Visualization Design and Validation.''} \emph{IEEE Transactions on Visualization and Computer Graphics} 15 (6): 921--28. \url{https://doi.org/10.1109/TVCG.2009.111}.

\end{CSLReferences}

\bibliography{RJreferences.bib}

\address{%
Krisanat Anukarnsakulchularp\\
Monash University\\%
Faculty of Business and Economics\\ Melbourne, Australia\\
%
%
\textit{ORCiD: \href{https://orcid.org/0009-0008-5638-7124}{0009-0008-5638-7124}}\\%
\href{mailto:kanu0003@student.monash.edu}{\nolinkurl{kanu0003@student.monash.edu}}%
}
